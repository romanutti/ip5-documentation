\chapter{Diskussion und Ausblick}

Das Erarbeiten des Link-Prediction-Plugins war eine herausfordernde und zugleich lehrreiche Arbeit.
Mit dem Projekt einher ging die Möglichkeit, im Open-Source-Umfeld zu einer etablierten Software beizutragen.
Insbesondere das modulübergreifende Anwenden von Wissen aus dem Studiengang und der Austausch im Team und mit dem Betreuer bereitete grosse Freude.
In den folgenden Kapiteln werden die Resultate reflektiert und ein Ausblick auf nachfolgende Tätigkeiten dargeboten.

\section{Erreichtes}

Das Ziel, ein Link-Prediction-Plugin zu entwickeln, ist erreicht worden. Der Pull-Request des Plugins wurde fristgerecht gestellt.
Bis zur Fertigstellung der Arbeit wurde der Pull-Request allerdings weder genehmigt, noch kommentiert.
Neben den erforderlichen Funktionen zur Vorhersage von Links und zur Evaluation der verschiedenen Algorithmen wurde zusätzlich ein Filter zum Selektieren der Kanten je Algorithmus entwickelt.
Die verwendete Architektur erlaubt es, einfach neue Link-Prediction-Algorithmen und Kennzahlen zur Beurteilung der Qualität der Algorithmen hinzuzufügen.
Als Reaktion auf die anfänglich langen Laufzeiten bei grossen Graphen wurde die Berechnung der Link Predictions optimiert:
Für die Berechnung der Link-Prediction-Werte müssen initial für sämtliche Knoten alle Knoten durchlaufen werden.
Indem beim Hinzufügen einer neuen Kante nur die noch möglichen Kanten der beiden veränderten Knoten neu berechnet werden, kann der Aufwand für alle weiteren Iterationen drastisch reduziert werden.

Tests mit der aktuellen Implementierung haben gezeigt, dass Link Predictions in Graphen mit bis zu 1000 Knoten problemlos und schnell berechnet werden können.
Das Berechnen in grösseren Graphen kann in längeren Laufzeiten resultieren.
Ein entsprechender Hinweis wurde im README.md vermerkt.

\section{Ausblick}

Trotz diverser Optimierungsbemühungen muss beim initialen Berechnen der Prediction-Werte die Knotenanzahl quadratisch durchlaufen werden.
Dies erfordert bei grossen Netzwerken seine Zeit.
Um auch bei solchen Netzwerken eine schnellere Performance zu erreichen, müssten weitere Optimierungsmassnahmen in Betracht gezogen werden.
Folgende Möglichkeiten würden sich dazu anbieten:

\begin{itemize}
    \item Anstatt, dass die Vorhersagen für alle Knoten berechnet werden, könnte man im \acs{ui} die Möglichkeit anbieten, ein Subset bestehend aus spezifischen Knoten auszuwählen. Damit würde die Berechnung nur für alle relevanten Knoten durchgeführt.
    \item Falls ein Clustering zur Verfügung steht (entweder über ein Knoten-Attribut oder über eine berechnete Messgrösse), könnte man die Berechnung auf alle Knoten dieses Clusters beschränken.
    \item Die Berechnung der verschiedenen Algorithmen könnte parallelisiert durchgeführt werden. Aktuell wird der Graph während der Berechnung mit einem Write-Lock geschützt. Dieser müsste für die Parallelisierung unter Berücksichtigung allfälliger Nebeneffekte aufgehoben werden.
\end{itemize}

Aktuell werden vom Plugin nur ungerichtete und ungewichtete Graphen unterstützt.
Falls die Algorithmen auf gerichtete Graphen angewendet werden, werden ausgehende und eingehende Kanten zwischen denselben zwei Knoten als eine Kante betrachtet:
Ein Graph mit einer Kante von Knoten $A$ nach Knoten $C$ und einer Kante von Knoten $C$ nach Knoten $A$ würde von den Algorithmen als eine Kante zwischen den beiden Knoten interpretiert.
In künftigen Releases könnte zusätzlich die Funktionalität angeboten werden, ungerichtete und gerichtete Graphen unterschiedlich zu behandeln.

Weiter könnten auch Kantengewichte und sofern vorhanden andere Attribute mit in die Berechnungen einbezogen werden.
Aktuell werden diese ignoriert und neue Kanten nicht gewichtet. Mögliche Ansätze zu Algorithmen mit gerichteten Graphen zeigen \citeauthor{garcia-gasulla_pdf_2014} (\citeyear{garcia-gasulla_pdf_2014}, S. 5).

Darüber hinaus beschränkte sich das User-Feedback auf den Auftraggeber.
Über die Facebook-Gruppe von Gephi wurde das Plugin zwar verbreitet, doch auf die Bitte um Feedback ist keines der Gruppenmitglieder eingegangen.
Falls das Plugin ergänzt oder optimiert werden sollte, könnte weiteres Feedback Hinweise auf Verbesserungsmöglichkeiten liefern.
