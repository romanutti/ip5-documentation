\chapter{Diskussion und Ausblick}

Das Erarbeiten des Link-Prediction Plugins war eine herausfordernde und zugleich lehrreiche Arbeit.
Mit dem Projekt einher ging die Möglichkeit, im Open-Source-Umfeld zu einer etablierten Software beizutragen.
Insesondere das modulübergreifende Anwenden von Wissen aus dem Studiengang und der Austausch im Team und mit dem Betreuer bereitete grosse Freude.
In den folgenden Kapiteln werden die Resultate reflektiert und ein Ausblick auf nacholgende Tätigkeiten dargeboten.

\section{Erreichtes}

Das Ziel, ein Link-Prediction Plugin zu entwickeln, ist erreicht worden.
Neben den erforderlichen Funktionen zur Vorhersage von Links und zur Evaluation der verschiedenen Algorithmen wurde zusätzlich ein Filter zum selektieren der Kanten je Algorithmus entwickelt.
Die verwendete Architektur erlaubt es, einfach neue Link-Prediction Algorithmen und Kennzahlen zur Beurteilung der Qualität der Algorithmen hinzuzufügen.
Als Reaktion auf die anfänglich langen Laufzeiten bei grossen Graphen wurde die Berechnung der Link-Predictions optimiert:
Für die Berechnung der Link-Prediction Werte müssen sämtliche Knoten initial zweimal durchlaufen werden.
Indem beim Hinzufügen einer neuen Kante nur die beiden davon betroffenen Kanten geändert werden, kann der Aufwand für alle weiteren Iterationen drastisch reduziert werden.

Tests mit der aktuellen Implementierung haben gezeigt, das Link-Predictions in Graphen mit bis zu 1000 Knoten problemlos berechnet werden können.
Das Berechnen in grösseren Graphen kann in längeren Laufzeiten resultieren.
Ein entsprechender Hinweis wurde im README.md vermerkt.

%TODO: Irgendwo beschrieben, dass mehrere Durchläufe zu mehreren Iterationen führen können

\section{Ausblick}

Dummy text.
