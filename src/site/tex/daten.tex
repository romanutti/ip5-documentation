\chapter{Daten}
\label{daten}
Um die Korrektheit von Link Predictions testen zu können, und um die Güte der verschiedenen Algorithmen bestimmen zu können werden bestehende Netzwerke verwendet.
Es wurden dafür Datensets, welche von \acs{gephi} auf dem GitHub-Repository bereitgestellt werden, verwendet (vgl. \citeauthor{moll_datasets_2018} \citeyear{moll_datasets_2018}, online).

\section{Anforderungen}
Für den Aufbau der verschiednene Tests wird ein Netzwerk zu einem bestimmten Zeitpunkt $t$ benötigt.
Auf diesem werden anschliessend Link Predictions durchgeführt.
Der veränderte Graph wird anschliessend mit dem effektiven Graphen zum Zeitpunkt $t + n$ verglichen.
Dies bedingt, dass die Kanten mit einer Nummerierung oder einem Zeitstempel versehen sind.
Die Anforderung besteht nur für die Entwicklung und Charakterisierung der Evaluationskomponente.
Im Plugin werden Graphen unabhänging von der Reihenfolge hinzugefügter Kanten bewertet.

Bei allen Netzwerken handelt es sich um ungerichtete Graphen.
% TODO Begriffe Graph / Netzwerk abgrenzen?

\section{Datenaufbereitung}
Um die Korrektheit einzelner und die Güte mehrerer Vorhersagen am effektiven Graphen zu überprüfen, müssen die Datensets in Initial- und Validierungsdaten unterteilt werden.
Die Unterteilung wird anhand der Kanten gemacht. Dazu werden die Kanten aufsteigend sortiert (älteste Kanten zuerst).
Anschliessend werden alle Daten aus dem Initialset dem Validierungsset zugeordnet. Auf dem Initialset werden danach für die verschiedenen Durchläufe die neuesten 1\%, 5\%, 10\% und 50\% der Kanten gelöscht. Das Initialset bildet das Basisnetzwerk zum Zeitpunkt $t$.
Das Testdatenset entspricht dem effektiven Netzwerk zum Zeitpunkt $t + n$.

In der Tabelle \ref{dataset_info} sind die Informationen zu den einzelnen Datensets ersichtlich.

\begin{threeparttable}
    \centering
    \label{dataset_info}
    \caption{Informationen der verwendeten Datensets}
    \begin{tabular}{@{}lllr@{}}
        \toprule
        Datenset                        & Anzahl Knoten & Anzahl Kanten & Grösse \\ \midrule
        Hypertext Conference Network\tnote{1}                 & 113     & 2163 & 2.5 MB   \\
        College Messages Network\tnote{2} & 845    & 3166 & 1.2 MB     \\
        Sparrow Social Network\tnote{3}                 & 52     & 446 & 7 KB   \\
        Enron Employees Network\tnote{4}                 & 151     & 1583 & 1.4 MB   \\
        Tortoise Network\tnote{5}                       & 135   & 374 & 19 KB   \\ \bottomrule
    \end{tabular}
    \begin{tablenotes}[\flushleft]
        \footnotesize
        \item[1] Contact network during the Hypertext 2009 conference. Source: http://ociopatterns.org.
        \item[2] Private messages sent on an online social network at the University of California, Irvine. Users could search the network for others and then initiate conversation based on profile information. An edge (u, v, t) means that user u sent a private message to user v at time t. Source: https://snap.stanford.edu/data/CollegeMsg.html
        \item[3] Interaction network of sparrows between two points of time. Interaction is defined as birds within an approximately 5-metre radius. Source: https://bansallab.github.io/asnr/data.html
        \item[4] Social network that represents 151 Enron employees. In this network each exchange of at least 5 emails between any pair of agents across the entire time range (1998 to 2002) is considered as a link. Source: https://pdfs.semanticscholar.org/4d7d/8fe43f391a966e0e15b68e6509fe6b533540.pdf
        \item[5] Social networks of desert tortoises captured at several points of time. Source: https://bansallab.github.io/asnr/data.html
    \end{tablenotes}
\end{threeparttable}

\section{Charakteristiken der Netzwerke}
Um die Netzwerke beschreiben zu können, werden verschiedene Metriken berechnet.
Nach \citeauthor{gao_link_2015} (\citeyear{gao_link_2015}, S. 5) weisen die nachfolgenden vier Messgrössen eine besonders starke Korrelation zum gewählten Link Prediction Algorithmus auf.

\textbf{Global Cluster Coefficient (GCC)}

Der \acs{cc} zeigt das Mass der Cliquenbildung in einem Graphen (\citeauthor{michael_henninger_soziale_2018} \citeyear{michael_henninger_soziale_2018}, S. 58).
Unter \acs{clique}n wird ein Subgraph innerhalb eines Graphen, in welchem die Knoten untereinander stark verbunden sind, verstanden.
In Cliquen stammt ein Grossteil der Informationen von denselben Informationslieferanten - dank der starken Verbundenheit kann sich die Gruppe schnell austauschen.

Der lokale Cluster Coeffizient wird aus dem Quotienten der Anzahl direkten Kanten $d$ zwischen den Nachbarn eines Knoten und der maximal möglichen Anzahl direkter Kanten $k_i(k_i -1)$ berechnet:

\begin{equation}
    \label{eq:ci}
    C_i = \frac{2*d}{k_i(k_i-1)}
\end{equation}


Bei ungerichteten Graphen wird die Anzahl direkter Kanten mit dem Faktor 2 multipliziert. $k_i$ steht für die Anzahl benachbarter Knoten.
Beim globalen Cluster Coeffizienten handelt es sich um den Mittelwert der lokalen Cluster Coefficienten.
In einem Graphen mit $N$ Knoten wird er folgendermassen berechnet:

\begin{equation}
    \label{eq:gcc}
    GCC = \frac{1}{N}\sum\limits_{i=1}^{N}C_i
% TODO Widerspruch zu Gao et. al
\end{equation}

\textbf{Density}

Die \acs{density} (dt. Dichte) eines Graphen zeigt an, wie komplett die Knoten des Netzwerkes miteinander verbunden sind.
Falls es in einem Netzwerk keine Kanten gibt, liegt der Density-Wert bei 0. Falls alle Knoten mit allen anderen Knoten verbunden sind, entspricht der Density-Wert 1.
Die Dichte ist in der Regel ein Indikator, wie schnell sich Informationen in einem Netzwerk verbreiten.
Die Density für ungerichtete Graphen berechnet sich mit unterstehender Gleichung, wobei $|E|$ für die Anzahl Kanten und $|V|$ für die Anzahl Knoten steht:
%TODO Formeln vereinheitlichen

\begin{equation}
    \label{eq:density}
    Density = \frac{2|E|}{|V|*(|V|-1)}
\end{equation}


Bei grösseren Netzwerken steigt mit der Anzahl Knoten auch die Anzahl möglicher Kanten.
Häufig ist zu beobachten, dass die Anzahl direkter Verbindungen dabei konstant bleibt - die Density nimmt bei grösseren Netzwerken folglich häufig ab (vgl. \citeauthor{michael_henninger_soziale_2018} \citeyear{michael_henninger_soziale_2018}, S. 56).

\textbf{Diameter}

Unter dem Graph \acs{diameter} (dt. Durchmesser) versteht man nach \citeauthor{michael_henninger_soziale_2018} (\citeyear{michael_henninger_soziale_2018}, S. 57) den \textit{längsten kürzesten Pfad} zwischen allen Knotenpaaren in einem Graphen.
Die Formel für den längsten kürzesten Pfad lautet folgendermassen:

\begin{equation}
    \label{eq:diameter}
    Diameter = max_i_jd(i,j)
\end{equation}

Wobei $d(i,j)$ der kürzeste Pfad zwischen den Knoten $i$ und $j$ ist.

\textbf{Average Shortest Path (ASP)}

Mit dem \acs{avgsp} (ASP) wird die durchschnittliche Länge des kürzesten Pfads zwischen zwei Knoten $i$ und $j$ berechnet (\citeauthor{gao_link_2015} \citeyear{gao_link_2015}, S. 6).
Die Formel dafür lautet folgendermassen:

\begin{equation}
    \label{eq:asp}
    ASP = \frac{1}{v * (v - 1)} * \sum\limits_{i!=j} d(i,j)
\end{equation}

Mithilfe dieser Metriken lässt sich die gesamte Netzwerkstruktur einordnen.
Tabelle \ref{tab_metrics} zeigt die berechneten Werte der verschiedenen Netzwerken.

\begin{table}[h]
    \centering
    %\resizebox{\textwidth}{!}{%
    \begin{tabular}{@{}llrll@{}}
        \toprule
        Datenset                         & GCC      & Density      & Diameter & ASP      \\ \midrule
        Hypertext Conference Network                & 0.535 & 0.342 & 3.000 & 1.661 \\
        College Messages Network         & 0.122 & 0.001 & 7.000 & 3.108 \\
        Sparrow Social Network           & 0.695 & 0.336 & 3.000 & 1.756 \\
        Enron Employees Network          & 0.519 & 0.133 & 4.000 & 2.136 \\
        Tortoise Network                 & 0.391 & 0.041 & 10.000 & 3.735 \\ \bottomrule
    \end{tabular}%
    %}
    \caption{Metrikwerte der Netzwerke}
    \label{tab_metrics}
\end{table}

Auffallend sind dabei insbesondere die tiefen Werte für \acs{gcc} und \acs{density} beim College Messages Network.
Daraus kann geschlossen werden, dass Knoten in diesem Netzwerk nur lose zusammenhängen und die Verknüpfung nicht sehr stark ausgeprägt ist.
Beim Tortoise Network weichen ausserdem die Werte für Density und Diameter von den Werten der anderen Datensets ab.
Die Werte legen die Vermutung nahe, dass dieses Netzwerk ebenfalls nicht sehr stark verknüpft ist.
% TODO Aussagen verifizieren
Diesen stark abweichenden Werten gilt bei der Interpration der erzielten Resultate der verschiedenen Algorithmen (vgl. Kapitel \ref{comparison}) besonderes Augenmerk.
