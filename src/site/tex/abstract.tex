% Set abstract title
\renewcommand*\abstractname{Summary}
\begin{abstract}
  Gephi ist eine Open-Source Software zur explorativen Analyse von Graphen.
  Die Software wird unter anderem im Modul \acl{sna} (\acs{sna}), welches an der Fachhochschule Nordwestschweiz unterrichtet wird, eingesetzt.
  Gephi ist modular aufgebaut und kann um eigene \acs{plugin}s erweitert werden.
  Um neue Kanten in einem bestehenden Netzwerk vorhersagen zu können, soll nun ein solches Plugin entwickelt werden.
  Die Vorhersage neuer Kanten (sog. ``Link Prediction'') kann mittels verschiedenen Algorithmen berechnet werden.
  In einer ersten Version wurden die Algorithmen ``Common Neighbours'' und ``Preferential Attachment'' implementiert.
  Der Aufbau des Plugins soll es erlauben, einfach neue Algorithmen hinzuzufügen.
  Nebst der Vorhersage neuer Kanten soll die Qualität der eingesetzten Link-Prediction-Algoithmen beurteilt werden können.
  Als Messgrösse wird hierfür die Accuracy verwendet, welche den prozentualen Anteil richtig vorausgesagter Kanten berechnet.

  Die Durchführbarkeit und Korrektheit der Konzepte wurde mittels einem Proof of Concept geprüft.
  Die daraus gewonnenen Erkenntnisse wurden in die anschliessende Umsetzung übernommen.
  % TODO Resultate Evaluation der verschiedenen Datensets

  Die erforderlichen Funktionalitäten des Plugins konnten umgesetzt werden.
  Das Plugin kann weiter verbessert werden, indem gerichtete und auch gewichtete Graphen unterstützt werden.
  Aktuell kann ausserdem nur für Netzwerken mit bis zu 1000 Knoten eine schnelle Laufzeit garantiert werden.
  Um auch bei grösseren Netzwerken eine schnellere Performance zu erreichen müssten weitere Optimierungsmassnahmen in Betracht gezogen werden.
\end{abstract}
