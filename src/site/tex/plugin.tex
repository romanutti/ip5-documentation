\chapter{Plugin}

Basierend auf den beschriebenen Erkenntnissen wurde ein Plugin für Gephi entwickelt, welches er erlaubt nächste Kanten mit verschiedenen Link Prediction Algorithmen vorherzusagen. Kanten, welche zum Netzwerk hinzugefügt worden sind, werden entsprechend gekennzeichnet. Die Anzahl vorhergesagter Kanten kann von den Benutzern angegeben werden.
Das Plugin enthält eine Evaluationskomponente, mit welcher die Qualität der verschiedenen Algorithmen verglichen werden kann.

Das Plugin wird unter der Apache 2.0 Lizenz veröffentlicht.
% TODO Add Gephi to glossary
% TODO Use cases in appendix

\section{Software Architektur}

Das Plugin besteht aus folgenden Hauptkomponenten:

\begin{itemize}
    \item \textbf{Statistik:} Neue Kanten können via Statistiken zu einem ungerichteten Graphen hinzugefügt werden. Für die ausgewählten Algorithmen werden die Link Preditction Werte berechnet und jeweils die Kante mit dem höchsten Wert zum Graphen hinzugefügt. Die Anzahl hinzuzufügender Kanten kann angegeben werden. Die einzelnen Kanten werden dabei iterativ hinzugefügt. Die Berechnung des nächstvorhergesagten Wertes basiert immer auf dem Graphen des vorherigen Iterationsschrittes.
    \item \textbf{Filter:} Die hinzugefügten Kanten können via Filter angezeigt werden. Dabei können die anzuzeigenden Kanten einerseits auf einen Filter, andererseits auf eine gewünschte Anzahl beschärnkt werden.
\end{itemize}

\subsection{Übersicht Klassen}
Die Klassenstruktur ist grösstenteils bereits durch das bestehende Grundgerüst von Gephi vorgegeben. Für eine möglichst einfache Erweiterbarkeit, werden entsprechende Design Patterns eingesetzt.

%TODO Refresh UMLs
Abbildung \ref{fig:class_statistic} zeigt das Klassendiagramm der Statistik-Klassen. Es basiert auf dem Command Pattern, um die Link Prediction Kennzahlen der verschiedenen Algorithmen zu berechnen. Das Makro Command Objekt wird eingesetzt um die Kennzahlen von mehreren ausgewählten Algorithmen zu berechnen.
%TODO Fussnote mit Link zu Command Pattern
\begin{figure}[h]
    \centering
    \includegraphics[width=\linewidth]{resources/class_Statistic.png}
    \caption{Klassendiagramm Statistiken}
    % TODO Add cite to SNA Script
    \label{fig:class_statistic}
\end{figure}

Die Filter-Klassen wurden mithilfe des Abstract Factory Patterns umgesetzt. Die Klassenstruktur besteht aus dem Factory Pattern mit verschiedenen Builder Klasse und dem Strategy Pattern, welches eine spezifische Umsetzung der Filter pro Algorithmus erlaubt. Abbildung \ref{fig:class_filter} zeigt das resultierende Klassendiagramm.
\begin{figure}[h]
    \centering
    \includegraphics[width=\linewidth]{resources/class_Filter.png}
    \caption{Klassendiagramm Filter}
    % TODO Add cite to SNA Script
    \label{fig:class_filter}
\end{figure}


\section{GUI Konzept}

Für das GUI Konzept wurde zuerst die Oberfläche von Gephi untersucht, damit die Link Prediction Funktionalität möglichst
nahtlos in diese integriert werden konnte. Hierbei stellten sich vor allem folgende Punkte heraus:

\begin{itemize}
    \item Unter dem Menü-Punkt "Statistics" werden diverse Werte des Graphen berechnet und gegebenenfalls im Data Laboratory
          hinzugefügt. Hier geschieht keinerlei Auswertung, Darstellung oder Einschränkung der Ergebniswerte.
    \item Unter dem Menü-Punkt "Filter" werden lediglich bereits vorhandene Daten, wie der Name bereits vermuten lässt,
          gefiltert. Hierbei werden Graphen für gewöhnlich verkleinert dargestellt, also mit weniger Kanten oder Knoten.
    \item Die Darstellung der verschiedenen Werte wird über den Menü-Punkt "Appearance" vorgenommen. Hier können Farben,
          und Grösse der Knoten und Kanten eingestellt werden.
    \item Im Data Laboratory sind die aktuellen Daten zu finden. Werden neue Daten über \textit{Statistics} berechnet
          oder die Daten im Laboratory verändert, werden nur die neusten Daten angezeigt. Um alte Daten zu erhalten,
          muss bei jeder Änderung ein neuer Workspace erstellt werden.
\end{itemize}

Mit Hilfe dieser Erkenntnisse kann die Benutzeroberfläche um das Link Prediction Plugin erweitert werden.

\subsection{Berechnung der neuen Link Prediction Kanten}

Die Berechnung der neuen Kanten muss im Bereich der \textit{Statistics} passieren, da hier neue Werte berechnet und dem
Data Laboratory hinzugefügt werden. Im Folgenden wird dieser Vorgang beschrieben:

\begin{figure}[htbp]
    \includegraphics[width=\linewidth]{resources/SC-1.png}
    \caption{Überblick der Ansicht Statistik.}
    \label{fig:screen1}
\end{figure}

Im Bild kann man eine Skizze sehen, welche einen Überblick über den Aufbau von Gephi gibt. Grau hinterlegt sind die
angewählten Menü-Punkte. Unter dem Menü \textit{Statistics} wird hier nun ein Punkt "Link Prediction" mit einem
Start-Button eingefügt. In der Darstellung ausgelassen wurden die anderen Punkte, die bereits in diesem Menü existieren
wie beispielsweise "Average Degree".

\begin{figure}[htbp]
    \includegraphics[width=\linewidth]{resources/SC-2.png}
    \caption{Pop-Up für die Auswahl der Berechnung.}
    \label{fig:screen2}
\end{figure}

Hier kann man sehen, dass beim Drücken auf "Start" ein Pop-Up geöffnet wird, in welchem man alle Algorithmen auswählen
kann, für welche die Link Prediction durchgeführt werden soll. In einem zusätzlichen Eingabefeld kann man einen Wert
von 1 bis x eingeben. Dieser besagt, wie viele Durchgänge bei der Link Prediction gemacht werden sollen - also wie viele
neue Kanten dem Graphen hinzugefügt werden.

Die Funktion der zusätzlichen Konfiguration eines Algorithmus ("advanced" Button) wird offen gehalten, wird aber für die
Implementierung des Common Neighbour und Preferential Attachment Algorithmus derzeit nicht benötigt.

Bei einer zu hohen Zahl oder der Auswahl bestimmter langwieriger Algorithmen soll eine Warnung an den User ausgegeben
werden. Wird diese bestätigt, so wird der Algorithmus trotzdem berechnet. Beim Abbruch kann der User seine gewünschten
Einstellungen verändern.

\begin{figure}[htbp]
    \includegraphics[width=\linewidth]{resources/SC-3.png}
    \caption{Modifizierter Graph mit vorhergesagtem Edge.}
    \label{fig:screen3}
\end{figure}

Im dritten Screen ist der Graph nach der Berechnung zu sehen. Es würden neue Kanten hinzugefügt. Diese können mit Hilfe
der Appearance Einstellungen eingefärbt werden, da im Data Laboratory Felder enthalten sind, die aussagen, ob eine
Kante mit einem Link Prediction Algorithmus hinzugefügt wurde oder ob sie bereits existiert hat.

\begin{figure}[htbp]
    \includegraphics[width=\linewidth]{resources/SC-4.png}
    \caption{Data Laboratory mit neuen Edges.}
    \label{fig:screen4}
\end{figure}

In diesem Bild kann man die durch die Algorithmen neu hinzugefügten Attribute im Data Laboratory sehen. Diese sind auf
den Kanten des Graphen zu finden. Dort gibt es ein Feld "Link Prediction Algorithm", in welchem ein Wert enthalten ist,
um eindeutig zu zeigen, mit welchem Algorithmus diese Kante hinzugefügt wurde.

Eine weitere Spalte ist "Added in Run". Gibt der User bei der Berechnung den Wert 3 mit, so werden hier 3 Kanten pro
ausgewähltem Algorithmus durchnummeriert hinzugefügt. Damit der User einen Überblick hat, in welchem Durchlauf diese
Kante hinzugefügt wurde, haben wir dieses Attribut hinzugefügt. 1 würde dabei für die am wahrscheinlichsten hinzugefügte
Kante im ersten Durchlauf stehen, 3 für die wahrscheinlichste im dritten Durchlauf, usw.

Bei Kanten, die bereits vorhanden sind, werden Default Werte für die Spalten "Link Prediction Algorithmus" und "Added
in Run" eingefügt.

Die letzte Spalte "Last Threshold Link Prediction" ist eine optionale Spalte, bei der wir noch weitere Analysen
durchführen müssen, um zu evaluieren, ob diese einen tieferen Sinn hat. Hier würde zum Beispiel der letzte berechnete
Wert einer neu hinzugefügten Kante eingetragen werden. Dann wäre diese im Fall von keiner Berechnung, da die Kante
bereits existierte, ebenfalls mit einem Default-Wert befüllt.

\subsection{Filtern der neuen Link Prediction Kanten}

Die neu errechneten Link Prediction Werte sollen gefiltert werden können. Auch dazu wurden einige Skizzen erstellt.

\begin{figure}[htbp]
    \includegraphics[width=\linewidth]{resources/SC-5.png}
    \caption{Überblick der Ansicht Filter.}
    \label{fig:screen5}
\end{figure}

Im Überblick ist zu sehen, wie der Link Prediction Filter in das bestehende GUI eingebunden wird. Dort wird es unter
\textit{Filter} einen neuen Ordner "Link Prediction" geben. Unter diesem können die Filter für einzelne Algorithmen
angewählt werden. Danach kann eingestellt werden, welche Links von welchen Durchläufen angezeigt werden sollen. Wird auf
"Start" geklickt, wird allenfalls das folgende Pop-Up angezeigt:

\begin{figure}[htbp]
    \includegraphics[width=\linewidth]{resources/SC-6.png}
    \caption{Fehlermeldung, wenn die Berechnung der Link Prediction noch nicht durchgeführt wurde.}
    \label{fig:screen6}
\end{figure}

Dieses Pop-Up stellt eine Fehlermeldung dar. Dort wird anstelle von "Link Prediction Filter" der ausgewählte Filter
stehen beispielsweise "Common Neighbour Filter" stehen. Diese Fehlermeldung erscheint, sofern die Berechnungen, um den
Filter anzuwenden, noch nicht ausgeführt wurden.

Optional behält sich das Projektteam vor, eine Option einzubauen, bei der direkt aus dem Fenster der Fehlermeldung die
Berechnung für diesen speziellen Algorithmus gestartet werden kann. Andernfalls kann die Fehlermeldung weggeklickt
werden und dann über die \textit{Statistics} die Berechnung vorgenommen werden.

\begin{figure}[htbp]
    \includegraphics[width=\linewidth]{resources/SC-7.png}
    \caption{Gefilterter Graph.}
    \label{fig:screen7}
\end{figure}

Wurde der Filter angewendet, so wird der Graph dargestellt mit den neu ergänzten Kanten. Die bereits vor der Berechnung
eingefügten Kanten werden auch dann angezeigt, egal wie hier gefiltert wird.

\subsection{Vergleich der Link Prediction Algorithmen mit tatsächlichen neuen Links}

Dummy text.

\section{Link Prediction}
Es gibt verschiedene Algorithmen, um neue Kanten vorherzusagen. Initial werden im Plugin die Algorithmen "Common Neighbours" und "Preferential Attachment" implementiert.
%Nach Recherche über die verschiedenen Algorithmen wurde entschieden, die Algorithmen "Common Neighbours" und
%"Preferential Attachment" zu implementieren. Diese Entscheidung beruhte darauf, dass sich die beiden Algorithmen
%insofern unterscheiden, dass beim Common-Neighbours-Algorithmus nur Knoten eine Rolle spielen, welche gemeinsame
%Nachbarn aufweisen. Beim zweiten Algorithmus ist dies anders: Dort werden alle Knoten im Bezug aufeinander
%berücksichtigt - hier ist also die Popularität eines einzelnen entscheidend, während beim ersteren Gemeinsamkeiten im
%Vordergrund stehen.

In der Literatur werden hauptsächlich Algorithmen beschrieben, welche auf ungerichtete Graphen anwendbar sind. Algorithmen für gerichtete Graphen weisen eine wesentlich höhere Komplexität auf. Die Arbeit beschränkt sich deshalb auf Link Predictions auf ungerichtete Graphen -
ein entsprechender Hinweis wurde im Gephi Plugin implementiert.

\subsection{Common Neighbours}
Der Common Neighbours Algorithmus berechnet für zwei nicht verbundene Knoten, wie viele gemeinsame Nachbarn existieren. Die Wahrscheinlichkeit
für das Entstehen einer neuen Kante ist umso höher, desto mehr gemeinsame Nachbarn zwei
Knoten haben. Common Neighbours beruht auf dem Social Forces Prinzip der Homophilie: Je grösser die Anzahl Gemeinsamkeiten in Form von Nachbarn, desto wahrscheinlicher wird eine neue Kante zwischen den beiden Knoten hinzugefügt (vgl. Kapitel \ref{socialforces}).

Die folgende Formel zeigt, wie die Anzahl der gemeinsamen Nachbarn von zwei Knoten $X$ und $Y$ berechnet werden kann::

$common\_neighbours(X,Y) = | N(X) \cap N(Y) |$

Der Wert, der aus der Rechnung für diesen Algorithmus resultiert, ist die Summe der Anzahl aller gemeinsamen
Nachbarn der beiden Knoten $X$ und $Y$. Dieser Wert muss für alle Knoten-Kombinationen berechnet und anschliessend der höchste
Wert ausgewählt werden.

Das Code-Listing \ref{lstCommonNeighbour} zeigt die vereinfachte Implementierung des Algorithmus.
Weil zweimal über sämtliche Knoten iteriert werden muss, besitzt die Implementierung die Komplexitätsklasse $O(n^2)$.

\begin{lstlisting}[caption={Common neighbour implementation},label=lstCommonNeighbour]
    // CommonNeighboursStatistics.java
    @Override public void execute(Graph graph) {

        //Iterate over all nodes
        for (Node a : graph.getNodes()) {

            // Remove self from nodes,
            // of which neighbourhood will be
            // verified
            List<Node> others = graph.getNodes().remove(a);

            // Get neighbours of a
            List<Node> aNeighbours = getNeighbours(graph, a);

            //Calculate common neighbours
            for (Node b : others) {

                // Get neighbours of b
                List<Node> bNeighbours = getNeighbours(graph, b);
                // Count number of common neighbours
                int commonNeighboursCount = getCommonNeighboursCount(aNeighbours, bNeighbours);

                // Temporary save calculated
                // value if edge does not exist
                if (isNewEdge()) {
                saveCalculatedValue(a, b, commonNeighboursCount)
                }
            }
        }

        // Add edge with highest calculation to graph
        Edge max = getHighestPrediction();
        graph.addEdge(max, "Common Neighbours);

    }
\end{lstlisting}

% TODO Describe path to statistics in GUI Section
In Gephi ist die Berechnung dieses Werts bei den Statistiken unter Link Prediction zu finden. Die berechneten Werte
werden anschliessend im Data Laboratory gespeichert und können mit Hilfe von Filtern eingeschränkt werden.

\begin{figure}[htbp]
    \includegraphics[width=\linewidth]{resources/gephi-CN.png}
    \caption{Gephi Common Neighbours Funktion.}
    \label{fig:screen8}
\end{figure}

\subsection{Preferential Attachment}
Mit dem Preferential Attachment Algorithmus werden ebenfalls die Nachbarn eines Knotens berücksichtigt. Diese müssen jedoch nicht
mit den Nachbarn des anderen Knoten übereinstimmen - es geht lediglich um die Anzahl der eigenen Nachbarn. Bei Preferential Attachment
wird von der Annahme ausgegangen, dass die Wahrscheinlichkeit einen neuen Kontakt zu knüpfen grösser, je grösser das eigene Netzwerk bereits ist.
Die Entstehung eines neuen Edges wird in diesem Fall nicht über Homophilie bestimmt, sondern über Ansehen (vgl. Kapitel \ref{socialforces}).

Die Formel für den Preferential Attachment Algorithmus lautet wie folgt:

$preferential_attachment(X,Y) = | N(X) | \cdot | N(Y ) |$

Um den resulierenden Wert zu erhalten, wie die Anzahl Nachbarn zweier Knoten $X$ und $Y$ miteinander multipliziert. Dieser Wert muss für alle Knoten-Kombinationen berechnet und anschliessend der höchste Wert
ausgewählt werden.

Der Algorithmus besitzt die Komplexitätsklasse $O(n^2)$. Das Code-Listing \ref{lstPreferentialAttachment} zeigt die vereinfachte Implementierung des Algorithmus.

\begin{lstlisting}[caption={Preferential attachment implementation},label=lstPreferentialAttachment]
    // PreferentialAttachmentStatistics.java
    @Override public void execute(Graph graph) {

        //Iterate over all nodes
        for (Node a : graph.getNodes()) {

            // Remove self from nodes,
            // of which neighbourhood will be
            // verified
            List<Node> others = graph.getNodes().remove(a);

            // Get neighbours of a
            List<Node> aNeighbours = getNeighbours(graph, a);

            //Calculate common neighbours
            for (Node b : others) {

                // Get neighbours of b
                List<Node> bNeighbours = getNeighbours(graph, b);
                // Multiply number of neighbours
                int totalNeighboursCount = aNeighbours.size() * bNeighbours.size();

                // Temporary save calculated
                // value if edge does not exist
                if (isNewEdge()) {
                    saveCalculatedValue(a, b, totalNeighboursCount)
                }
            }
        }

        // Add edge with highest calculation to graph
        Edge max = getHighestPrediction();
        graph.addEdge(max, "Preferential Attachment");

    }
\end{lstlisting}

In Gephi ist die Berechnung dieses Werts bei den Statistiken unter Link Prediction zu finden. Die berechneten Werte
werden anschliessend im Data Laboratory gespeichert und können mit Hilfe von Filtern eingeschränkt werden.

\begin{figure}[htbp]
    \includegraphics[width=\linewidth]{resources/gephi-PA.png}
    \caption{Gephi Preferential Attachment Funktion.}
    \label{fig:screen9}
\end{figure}

\section{Netzwerk- und Algorithmenvergleich}

Dummy text.
