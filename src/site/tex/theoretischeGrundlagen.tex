\chapter{Theoretische Grundlagen}

\section{Begriffsdefinitionen}
In den folgenden Kapiteln werden die Begriffe „Social Network Analysis“, „Graph“ und „Link-Prediction“ einzeln
erläutert. Es handelt sich in Bezug auf die Arbeit um grundlegende Begriffe. Weitere Fachbegriffe, die
in der Arbeit verwendet werden, sind im Glossar beschrieben.
% TODO Make glossary

\cite{gao_link_2015}
\subsection{Social Network Analysis}
Soziale Netzwerke bestehen aus Akteuren, wie beispielsweise Individuen, Organisationen oder ganze Nationen und deren Beziehungen.\cite{noauthor_soziale_nodate}.
Unter \acl{sna} (SNA, dt. Soziale Netzwerkanalyse) wird die Methode zur Erfassung und Analyse von dieser Netzwerke und der Beziehungen darin verstanden \cite{noauthor_soziale_2019}.
Die soziale Netzwerkanalyse ist ein interdisziplinäres Forschungsfeld und wird in den Sozial- und Verhaltenswissenschaften aber auch in der Betriebswirtschaftslehre oder den Politikwissenschafften angewandt \cite{noauthor_soziale_nodate}.
Gegenwärtig sind sogenannte "soziale Netzwerke" allgegenwärtig und werden insbesondere im Kontext von Internet-Plattformen wie \textit{Facebook} oder \textit{Twitter} häufig betrachtet.

\subsection{Graph}
Unter
\subsection{Link-Prediction}

Dummy text.
