\chapter{Einleitung}
\section{Ausgangslage}
Dieses Dokument dient der Beschreibung des Projekts \textit{Link-Prediction Plugin für Gephi (I4DS01)}.
Gephi ist eine Open-Source Software zur explorativen Analyse von Graphen. Die Software ist in der Programmiersprache Java implementiert und modular aufgebaut. Durch den modularen Aufbau können individuelle Plugins entwickelt werden. Gephi bietet eine Architektur, welche es erlaubt, die bestehende Funktionalität einfach um solche eigenen Plugins zu erweitern.

Eine Liste der aktuell verfügbaren Plugins ist einsehbar unter \href{https://gephi.org/plugins}{gepi.org/plugins}.

\section{Zielsetzung}

Das Ziel dieser Arbeit ist es, ein Link-Prediction Plugin für Gephi zu entwickeln. Unter dem Begriff "Link-Prediction" wird die Vorhersage der nächsten Kanten in einem bestehenden Netztwerk aus mehreren Knoten verstanden.
Der Funktionsumfang des Plugins beschränkt sich auf folgende Hauptanforderungen:

\begin{itemize}
    \item Auf einem bestehenden Netzwerk können mittels verschiedenen Link-Prediction Algorithmen die Kanten vorausgesagt werden, welche sich als nächstes verbinden. Kanten, welche aufgrund der Link-Prediction zum Netzwerk hinzugefügt werden, müssen entsprechend gekennzeichnet werden.
    \item Die Qualität der eingesetzten Link-Prediction-Algorithmen für durchgeführte Vorhersagen kann bewertet werden. Dadurch ist es möglich, für ein gegebenes Netzwerk den bestgeeigneten Algorithmus zu evaluieren.
\end{itemize}

Bei der Architektur des Plugins ist insbesondere wichtig, dass es einfach um weitere Link-Prediction Algorithmen erweitert werden kann.
Die genaue Aufgabenstellung ist unter Anhang \ref{projektvereinbarung} zu finden.

\section{Verwendete Software und Lizenzen}

Nachfolgend werden die wichtigsten Komponenten beschrieben, welche für das Entwickeln und Builden des Gephi-Plugins eingesetzt wurden.

\begin{itemize}
    \item \textbf{Gephi:} Grundlage des Link-Prediction Plugins bildet die Software Gephi, welche zur Visualisierung und Analyse von Graphen und Netzwerken eingesetzt wird. Das Plugin wurde für Gephi Version 0.9.2 entwickelt. Bei Gephi handelt es sich um eine kostenlose Open-Source-Software. % TODO Lizenz
    \item \textbf{Netbeans:} Für die Implementation in Java wurde grösstenteils die Entwicklungsumgebung NetBeans eingesetzt. Gegenüber anderen Entwicklungsumgebungen bietet NetBeans beim Einsatz mit Gephi Vorteile beim Debugging und beim Erstellen von grafischen Benutzeroberflächen.
    \item \textbf{Github:} Als Code-Repository wurde GitHub\footnote{http://github.com/} eingesetzt.
    \item \textbf{Travis CI:} Das Pluing wurde als Maven-Projekt aufgesetzt. Die Software wird mittels Travis CI\footnote{http://travis-ci.org/} gebuildet und getestet. Weil es sich bei dem Plugins ebenfalls um ein Open-Source-Projekt handelt, kann die Cloud von Travis kostenlos genutzt werden.
    \item \textbf{Sonarcloud:} Neben der statischen Code-Analyse wird der entwickelte Code mittels Sonarcloud\footnote{http://sonarcloud.io/} analysiert und geprüft. Die Einbindung des Projekts in Sonarcloud ist für Open-Source-Projekte ebenfalls kostenlos.
\end{itemize}
