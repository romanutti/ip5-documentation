\chapter*{Glossar}

\begin{acronym}
    %TODO: Pfad, Knoten, Kante
    \acro{avgsp}[Average Shortest Path]{Durchschnittliche Länge des kürzesten Pfads zwischen zwei Knoten i und j.}
    \acro{clique}[Clique]{Subgraph innerhalb eines Graphen, in welchem die Knoten untereinander stark verbunden sind.}
    \acro{cc}[Cluster Coefficient] Messgrösse, welche das Mass der Cliquenbildung in einem Graphen aufzeigt.
    \acro{datalaboratory}[Data Laboratory]{Knoten- und Kantentabellen eines Graphen in Gephi.}
    \acro{density}[Density]{Messgrösse, welche verdeutlicht, wie komplett die Knotten des Netzwerks miteinander verbunden sind.}
    \acro{diameter}[Diameter]{Messgrösse, welche den längsten kürzesten Pfad zwischen allen Knottenpaaren in einem Graphen entspricht.}
    \acro{gephi}[Gephi]{Gephi ist eine Open-Source Software zur explorativen
    Analyse von Graphen. Die Software ist in der Programmiersprache Java implementiert
    und modular aufgebaut.}
    \acro{plugin}[Plugin]{Ein Plug-in [ˈplʌgɪn] (häufig auch Plugin; von engl. to plug in, „einstöpseln, anschließen“, auch Software-Erweiterung oder Zusatzmodul) ist eine optionale Software-Komponente, die eine bestehende Software oder Computerspiel erweitert bzw. verändert (\cite{noauthor_plug-_2019}).}
\end{acronym}
