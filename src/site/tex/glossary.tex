\chapter*{Glossar}

\begin{acronym}
    %TODO: Pfad, Knoten, Kante
    \acro{avgsp}[Average Shortest Path]{Messgrösse zur Charakterisierung eines Netzwerkes, entsprich der durchschnittlichen Länge des kürzesten Pfads zwischen zwei Knoten.}
    \acro{clique}[Clique]{Subgraph innerhalb eines Graphen, in welchem die Knoten untereinander stark verbunden sind.}
    \acro{cc}[Cluster Coefficient]{Messgrösse zur Charakterisierung von Netzwerken, welche das Mass der Cliquenbildung in einem Graphen aufzeigt.}
    \acro{datalaboratory}[Data Laboratory]{Knoten- und Kantentabellen eines Graphen in Gephi. Das Data Laboratory (dt. Datenlabor) kann über den gleichnahmigen Button im Gephi-UI geöffnet werden.}
    \acro{density}[Density]{Messgrösse, welche verdeutlicht, wie komplett die Knotten des Netzwerks miteinander verbunden sind.}
    \acro{diameter}[Diameter]{Messgrösse, welche dem längsten kürzesten Pfad zwischen allen Knottenpaaren in einem Graphen entspricht.}
    \acro{gephi}[Gephi]{Gephi ist eine Open-Source Software zur explorativen
    Analyse von Graphen. Die Software ist in der Programmiersprache Java implementiert
    und modular aufgebaut. Für das Projekt wurde die Version 0.9.2 eingesetzt.}
    \acro{plugin}[Plugin]{Ein Plugin ist eine optionale Software-Komponente, die eine bestehende Software erweitert bzw. verändert (\cite{noauthor_plug-_2019}).}
\end{acronym}
