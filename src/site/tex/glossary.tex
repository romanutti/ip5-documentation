\chapter*{Glossar}
\addcontentsline{toc}{chapter}{Glossar}

\begin{acronym}
    \acro{abstractfactory}[Abstract Factory Pattern]{Die abstrakte Fabrik (engl. abstract factory) ist ein Entwurfsmuster aus dem Bereich der Softwareentwicklung, das zur Kategorie der Erzeugungsmuster gehört. Es definiert eine Schnittstelle zur Erzeugung einer Familie von Objekten, wobei die konkreten Klassen der zu instanziierenden Objekte nicht näher festgelegt werden (vgl. \citeauthor{wikipedia_abstrakte_2018} \citeyear{wikipedia_abstrakte_2018}, online).}
    \acro{avgsp}[Average Shortest Path]{Messgrösse zur Charakterisierung eines Netzwerkes, entspricht der durchschnittlichen Länge des kürzesten Pfads zwischen zwei Knoten.}
    \acro{clique}[Clique]{Subgraph innerhalb eines Graphen, in welchem die Knoten untereinander stark verbunden sind.}
    \acro{cc}[Cluster Coefficient]{Messgrösse zur Charakterisierung von Netzwerken, welche das Mass der Cliquenbildung in einem Graphen aufzeigt.}
    \acro{compositecommand}[Composite Command Pattern]{Das Composite Pattern beruht darauf, in einer abstrakten Klasse sowohl konkrete Objekte als auch ihre Behälter zu repräsentieren. Damit können sowohl einzelne Objekte als auch ihre Kompositionen einheitlich behandelt werden (\citeauthor{wikipedia_kompositum_2019} \citeyear{wikipedia_kompositum_2019}, online). Beim Composite Command Pattern fungiert die Kompositionsklasse als Makro-Klasse. Sie enthält eine Liste von Objekten, welche einzeln ausgeführt werden können.}
    \acro{datalaboratory}[Data Laboratory]{Knoten- und Kantentabellen eines Graphen in Gephi. Das Data Laboratory (dt. Datenlabor) kann über den gleichnamigen Button im Gephi-UI geöffnet werden.}
    \acro{density}[Density]{Messgrösse, welche verdeutlicht, wie komplett die Knoten des Netzwerks miteinander verbunden sind.}
    \acro{diameter}[Diameter]{Messgrösse, welche dem längsten kürzesten Pfad zwischen allen Knotenpaaren in einem Graphen entspricht.}
    \acro{gephi}[Gephi]{Gephi ist eine Open-Source-Software zur explorativen
    Analyse von Graphen. Die Software ist in der Programmiersprache Java implementiert
    und modular aufgebaut. Für das Projekt wurde die Version 0.9.2 eingesetzt.}
    \acro{kante}[Kante]{Verbindung zwischen zwei Knoten, die die Interaktion zwischen den Knoten darstellt.}
    \acro{knoten}[Knoten]{Aktoren in einem Netzwerk.}
    \acro{pfad}[Pfad]{Ein Pfad bezeichnet die Kanten, welche genutzt werden müssen, um von einem Knoten zu einem anderen zu gelangen.}
    \acro{plugin}[Plugin]{Ein Plugin ist eine optionale Software-Komponente, die eine bestehende Software erweitert bzw. verändert (vgl. \citeauthor{wikipedia_plug-_2019} \citeyear{wikipedia_plug-_2019}, online).}
    \acro{socialforces}[Social Forces]{Motivator für Personen soziale Beziehungen einzugehen.}
    \acro{templatepattern}[Template Pattern]{Das Template Pattern ist ein in der Softwareentwicklung eingesetztes Entwurfsmuster, mit dem Teilschritte eines Algorithmus variabel gehalten werden können während dabei eine Grundstruktur vorgegeben wird. Beim Template Pattern wird in einer abstrakten Klasse das Skelett eines Algorithmus definiert. Die konkrete Ausformung der einzelnen Schritte wird an Unterklassen delegiert. Dadurch besteht die Möglichkeit, einzelne Schritte des Algorithmus zu verändern oder zu überschreiben, ohne dass die zu Grunde liegende Struktur des Algorithmus modifiziert werden muss (\citeauthor{wikipedia_schablonenmethode_2018} \citeyear{wikipedia_schablonenmethode_2018}, online).}
\end{acronym}
